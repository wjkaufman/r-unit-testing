\documentclass{beamer}

\usepackage{hyperref}
\usepackage{color}
\usepackage{listings}

\hypersetup{colorlinks = true, urlcolor = blue}

\title{Unit Testing in R}
\author{Will Kaufman}
\date{\today}

\begin{document}

\frame{\titlepage}

%\section[Outline]{}
%\frame{\tableofcontents}

\section{Introduction to Unit Testing}
%%%
\begin{frame}
\frametitle{What is Unit Testing?}

\begin{itemize}
\item ``Unit testing is a software testing method by which individual units of source code\ldots are tested to determine whether they are fit for use."\footnote{\url{https://en.wikipedia.org/wiki/Unit_testing}}
\item Allows for more efficient development of code and packages.
\item Tests for each basic principle of the code, multiple situations and incorrect invocations.
\item Newly discovered bugs and errors are added as additional test cases. 
\end{itemize}
\end{frame}

\section{Unit Testing in R}
%%%
\begin{frame}
	\frametitle{Options for Unit Testing in R}
	
	\begin{itemize}
	\item RUnit by Matthias Burger, Klaus Juenemann, and Thomas Koenig.\footnote{\url{https://cran.r-project.org/web/packages/RUnit/index.html}.}
	\item testthat by Hadley Wickham.\footnote{\url{https://cran.r-project.org/web/packages/testthat/index.html}.}
	\item RUnit implements the same syntax as other unit testing implementations in other languages (JUnit, CppUnit, PerlUnit). This tutorial will focus on RUnit (though both are acceptable solutions).
	\end{itemize}
\end{frame}

\subsection{RUnit Package}
%%%
\begin{frame}

\frametitle{RUnit Documentation and Resources}

\begin{itemize}
\item For more detailed explanations and examples, see \href{https://cran.r-project.org/web/packages/RUnit/RUnit.pdf}{the reference manual} and \href{https://cran.r-project.org/web/packages/RUnit/vignettes/RUnit.pdf}{the package vignette}.
\item For a short walkthrough of RUnit basics, see \href{http://www.johnmyleswhite.com/notebook/2010/08/17/unit-testing-in-r-the-bare-minimum/}{this tutorial by John White}.
\item For other questions, Google is your friend.
\end{itemize}

\end{frame}

%%%
\begin{frame}[fragile]
\frametitle{Creating Unit Tests}

\begin{itemize}
\item Given a package 
\item
\end{itemize}

% checkFuncs
\begin{lstlisting}[language=R, numbers = left]
checkEquals(target, current, msg,
	tolerance = .Machine$double.eps^0.5,
	checkNames = TRUE, ...)
checkEqualsNumeric(target, current, msg,
	tolerance = .Machine$double.eps^0.5, ...)
checkIdentical(target, current, msg)
checkTrue(expr, msg)
checkException(expr, msg,
	silent = getOption("RUnit")$silent)
DEACTIVATED(msg)
\end{lstlisting}
\end{frame}

\begin{frame}
\frametitle{Tests Involving Random Numbers}

\begin{itemize}
\item Specify the type of random number generator in \texttt{defineTestSuite}, and specify \texttt{set.seed} in the test function
\end{itemize}

\end{frame}

\end{document}

